%
% API Documentation for GroundSegment Technical DataSheet
% Module GroundSegment.models.Notification.Notification
%
% Generated by epydoc 3.0.1
% [Tue Sep 27 17:02:06 2016]
%

%%%%%%%%%%%%%%%%%%%%%%%%%%%%%%%%%%%%%%%%%%%%%%%%%%%%%%%%%%%%%%%%%%%%%%%%%%%
%%                          Module Description                           %%
%%%%%%%%%%%%%%%%%%%%%%%%%%%%%%%%%%%%%%%%%%%%%%%%%%%%%%%%%%%%%%%%%%%%%%%%%%%

    \index{GroundSegment \textit{(package)}!GroundSegment.models \textit{(package)}!GroundSegment.models.Notification \textit{(package)}!GroundSegment.models.Notification.Notification \textit{(module)}|(}
\section{Module GroundSegment.models.Notification.Notification}

    \label{GroundSegment:models:Notification:Notification}
\begin{alltt}

Created on Sep 27, 2016

@author: ubuntumate
\end{alltt}


%%%%%%%%%%%%%%%%%%%%%%%%%%%%%%%%%%%%%%%%%%%%%%%%%%%%%%%%%%%%%%%%%%%%%%%%%%%
%%                           Class Description                           %%
%%%%%%%%%%%%%%%%%%%%%%%%%%%%%%%%%%%%%%%%%%%%%%%%%%%%%%%%%%%%%%%%%%%%%%%%%%%

    \index{GroundSegment.models.Notification.Notification.Notification \textit{(class)}|(}
\subsection{Class Notification}

    \label{GroundSegment:models:Notification:Notification:Notification}
\begin{tabular}{cccccc}
% Line for django.db.models.Model, linespec=[False]
\multicolumn{2}{r}{\settowidth{\BCL}{django.db.models.Model}\multirow{2}{\BCL}{django.db.models.Model}}
&&
  \\\cline{3-3}
  &&\multicolumn{1}{c|}{}
&&
  \\
&&\multicolumn{2}{l}{\textbf{GroundSegment.models.Notification.Notification.Notification}}
\end{tabular}

\begin{alltt}

Notifacion efectiva enviada o por enviar. Generada a partir de la configuracion del sistema

Objetivo
=========
    Repositorio para las notificaciones generadas y enviadas.

 
Implementacion
=========
    Se utiliza una unica entidad, las notificaciones se marcan como envidas cuando pueden ser despachadas o se
    indican la cantidad de intentos y se marcan como fallidas
\end{alltt}


%%%%%%%%%%%%%%%%%%%%%%%%%%%%%%%%%%%%%%%%%%%%%%%%%%%%%%%%%%%%%%%%%%%%%%%%%%%
%%                            Class Variables                            %%
%%%%%%%%%%%%%%%%%%%%%%%%%%%%%%%%%%%%%%%%%%%%%%%%%%%%%%%%%%%%%%%%%%%%%%%%%%%

  \subsubsection{Class Variables}

    \vspace{-1cm}
\hspace{\varindent}\begin{longtable}{|p{\varnamewidth}|p{\vardescrwidth}|l}
\cline{1-2}
\cline{1-2} \centering \textbf{Name} & \centering \textbf{Description}& \\
\cline{1-2}
\endhead\cline{1-2}\multicolumn{3}{r}{\small\textit{continued on next page}}\\\endfoot\cline{1-2}
\endlastfoot\raggedright a\-l\-a\-r\-m\- & \raggedright \begin{alltt}

Alarma asociada a la notificacion, puede no existir si la notificacion fuera generada por otro evento

distinto a la generacion de la alarma
\end{alltt}

\textbf{Value:} 
{\tt models.ForeignKey(Alarm, related\_name= "notifications", o\texttt{...}}&\\
\cline{1-2}
\raggedright t\-e\-x\-t\- & \raggedright \textbf{Value:} 
{\tt models.TextField(null= False, default= "Sin mensaje")}&\\
\cline{1-2}
\end{longtable}

    \index{GroundSegment.models.Notification.Notification.Notification \textit{(class)}|)}
    \index{GroundSegment \textit{(package)}!GroundSegment.models \textit{(package)}!GroundSegment.models.Notification \textit{(package)}!GroundSegment.models.Notification.Notification \textit{(module)}|)}
