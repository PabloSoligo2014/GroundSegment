%
% API Documentation for GroundSegment Technical DataSheet
% Module GroundSegment.models.Satellite
%
% Generated by epydoc 3.0.1
% [Tue Dec 13 14:58:02 2016]
%

%%%%%%%%%%%%%%%%%%%%%%%%%%%%%%%%%%%%%%%%%%%%%%%%%%%%%%%%%%%%%%%%%%%%%%%%%%%
%%                          Module Description                           %%
%%%%%%%%%%%%%%%%%%%%%%%%%%%%%%%%%%%%%%%%%%%%%%%%%%%%%%%%%%%%%%%%%%%%%%%%%%%

    \index{GroundSegment \textit{(package)}!GroundSegment.models \textit{(package)}!GroundSegment.models.Satellite \textit{(module)}|(}
\section{Module GroundSegment.models.Satellite}

    \label{GroundSegment:models:Satellite}
Created on 16 de ago. de 2016

\textbf{Author:} pablo soligo




%%%%%%%%%%%%%%%%%%%%%%%%%%%%%%%%%%%%%%%%%%%%%%%%%%%%%%%%%%%%%%%%%%%%%%%%%%%
%%                           Class Description                           %%
%%%%%%%%%%%%%%%%%%%%%%%%%%%%%%%%%%%%%%%%%%%%%%%%%%%%%%%%%%%%%%%%%%%%%%%%%%%

    \index{GroundSegment.models.Satellite.Satellite \textit{(class)}|(}
\subsection{Class Satellite}

    \label{GroundSegment:models:Satellite:Satellite}
\begin{tabular}{cccccc}
% Line for django.db.models.Model, linespec=[False]
\multicolumn{2}{r}{\settowidth{\BCL}{django.db.models.Model}\multirow{2}{\BCL}{django.db.models.Model}}
&&
  \\\cline{3-3}
  &&\multicolumn{1}{c|}{}
&&
  \\
&&\multicolumn{2}{l}{\textbf{GroundSegment.models.Satellite.Satellite}}
\end{tabular}

Clase/Entidad Satelite.


%%%%%%%%%%%%%%%%%%%%%%%%%%%%%%%%%%%%%%%%%%%%%%%%%%%%%%%%%%%%%%%%%%%%%%%%%%%
%%                                Methods                                %%
%%%%%%%%%%%%%%%%%%%%%%%%%%%%%%%%%%%%%%%%%%%%%%%%%%%%%%%%%%%%%%%%%%%%%%%%%%%

  \subsubsection{Methods}

    \label{GroundSegment:models:Satellite:Satellite:new}
    \index{GroundSegment.models.Satellite.Satellite \textit{(class)}!GroundSegment.models.Satellite.Satellite.new \textit{(class method)}}

    \vspace{0.5ex}

\hspace{.8\funcindent}\begin{boxedminipage}{\funcwidth}

    \raggedright \textbf{new}(\textit{cls}, \textit{code}, \textit{description}, \textit{noradId})

    \vspace{-1.5ex}

    \rule{\textwidth}{0.5\fboxrule}
\setlength{\parskip}{2ex}
    Constructor de clase

\setlength{\parskip}{1ex}
      \textbf{Return Value}
    \vspace{-1ex}

      \begin{quote}
      Nueva instancia del satelite

      {\it (type=Satellite)}

      \end{quote}

    \end{boxedminipage}

    \label{GroundSegment:models:Satellite:Satellite:getCode}
    \index{GroundSegment.models.Satellite.Satellite \textit{(class)}!GroundSegment.models.Satellite.Satellite.getCode \textit{(method)}}

    \vspace{0.5ex}

\hspace{.8\funcindent}\begin{boxedminipage}{\funcwidth}

    \raggedright \textbf{getCode}(\textit{self})

\setlength{\parskip}{2ex}
\setlength{\parskip}{1ex}
    \end{boxedminipage}

    \label{GroundSegment:models:Satellite:Satellite:getLastTLE}
    \index{GroundSegment.models.Satellite.Satellite \textit{(class)}!GroundSegment.models.Satellite.Satellite.getLastTLE \textit{(method)}}

    \vspace{0.5ex}

\hspace{.8\funcindent}\begin{boxedminipage}{\funcwidth}

    \raggedright \textbf{getLastTLE}(\textit{self})

    \vspace{-1.5ex}

    \rule{\textwidth}{0.5\fboxrule}
\setlength{\parskip}{2ex}
    Verificar la fecha de ultima descarga del TLE, si puede existir un TLE 
    nuevo intentar descargarlo

\setlength{\parskip}{1ex}
    \end{boxedminipage}

    \label{GroundSegment:models:Satellite:Satellite:getCelestialPosition}
    \index{GroundSegment.models.Satellite.Satellite \textit{(class)}!GroundSegment.models.Satellite.Satellite.getCelestialPosition \textit{(method)}}

    \vspace{0.5ex}

\hspace{.8\funcindent}\begin{boxedminipage}{\funcwidth}

    \raggedright \textbf{getCelestialPosition}(\textit{self}, \textit{dtm}={\tt datetime.now(utc)})

\setlength{\parskip}{2ex}
\setlength{\parskip}{1ex}
    \end{boxedminipage}

    \label{GroundSegment:models:Satellite:Satellite:newAlarm}
    \index{GroundSegment.models.Satellite.Satellite \textit{(class)}!GroundSegment.models.Satellite.Satellite.newAlarm \textit{(method)}}

    \vspace{0.5ex}

\hspace{.8\funcindent}\begin{boxedminipage}{\funcwidth}

    \raggedright \textbf{newAlarm}(\textit{self}, \textit{alarmType})

\setlength{\parskip}{2ex}
\setlength{\parskip}{1ex}
    \end{boxedminipage}

    \label{GroundSegment:models:Satellite:Satellite:__str__}
    \index{GroundSegment.models.Satellite.Satellite \textit{(class)}!GroundSegment.models.Satellite.Satellite.\_\_str\_\_ \textit{(method)}}

    \vspace{0.5ex}

\hspace{.8\funcindent}\begin{boxedminipage}{\funcwidth}

    \raggedright \textbf{\_\_str\_\_}(\textit{self})

\setlength{\parskip}{2ex}
\setlength{\parskip}{1ex}
    \end{boxedminipage}


%%%%%%%%%%%%%%%%%%%%%%%%%%%%%%%%%%%%%%%%%%%%%%%%%%%%%%%%%%%%%%%%%%%%%%%%%%%
%%                            Class Variables                            %%
%%%%%%%%%%%%%%%%%%%%%%%%%%%%%%%%%%%%%%%%%%%%%%%%%%%%%%%%%%%%%%%%%%%%%%%%%%%

  \subsubsection{Class Variables}

    \vspace{-1cm}
\hspace{\varindent}\begin{longtable}{|p{\varnamewidth}|p{\vardescrwidth}|l}
\cline{1-2}
\cline{1-2} \centering \textbf{Name} & \centering \textbf{Description}& \\
\cline{1-2}
\endhead\cline{1-2}\multicolumn{3}{r}{\small\textit{continued on next page}}\\\endfoot\cline{1-2}
\endlastfoot\raggedright c\-o\-d\-e\- & \raggedright \textbf{Value:} 
{\tt models.CharField('Codigo del satelite', max\_length= 24, h\texttt{...}}&\\
\cline{1-2}
\raggedright d\-e\-s\-c\-r\-i\-p\-t\-i\-o\-n\- & \raggedright \textbf{Value:} 
{\tt models.CharField('Decripcion del satelite', max\_length= 1\texttt{...}}&\\
\cline{1-2}
\raggedright n\-o\-r\-a\-d\-I\-d\- & \raggedright \textbf{Value:} 
{\tt models.IntegerField('Codigo norad del satelite', help\_tex\texttt{...}}&\\
\cline{1-2}
\raggedright a\-c\-t\-i\-v\-e\- & \raggedright \textbf{Value:} 
{\tt models.BooleanField('Activacion/desactivacion del satelit\texttt{...}}&\\
\cline{1-2}
\raggedright s\-t\-a\-t\-e\- & \raggedright \textbf{Value:} 
{\tt models.ForeignKey(SatelliteState, related\_name= 'satellit\texttt{...}}&\\
\cline{1-2}
\raggedright n\-o\-t\-e\-s\- & \raggedright \textbf{Value:} 
{\tt models.TextField('Observaciones sobre el satelite', max\_l\texttt{...}}&\\
\cline{1-2}
\end{longtable}

    \index{GroundSegment.models.Satellite.Satellite \textit{(class)}|)}
    \index{GroundSegment \textit{(package)}!GroundSegment.models \textit{(package)}!GroundSegment.models.Satellite \textit{(module)}|)}
