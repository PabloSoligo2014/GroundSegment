%
% API Documentation for GroundSegment Technical DataSheet
% Module GroundSegment.models.Tle
%
% Generated by epydoc 3.0.1
% [Tue Sep 27 17:02:06 2016]
%

%%%%%%%%%%%%%%%%%%%%%%%%%%%%%%%%%%%%%%%%%%%%%%%%%%%%%%%%%%%%%%%%%%%%%%%%%%%
%%                          Module Description                           %%
%%%%%%%%%%%%%%%%%%%%%%%%%%%%%%%%%%%%%%%%%%%%%%%%%%%%%%%%%%%%%%%%%%%%%%%%%%%

    \index{GroundSegment \textit{(package)}!GroundSegment.models \textit{(package)}!GroundSegment.models.Tle \textit{(module)}|(}
\section{Module GroundSegment.models.Tle}

    \label{GroundSegment:models:Tle}
\begin{alltt}

Created on Aug 24, 2016

@author: Pablo Soligo
\end{alltt}


%%%%%%%%%%%%%%%%%%%%%%%%%%%%%%%%%%%%%%%%%%%%%%%%%%%%%%%%%%%%%%%%%%%%%%%%%%%
%%                           Class Description                           %%
%%%%%%%%%%%%%%%%%%%%%%%%%%%%%%%%%%%%%%%%%%%%%%%%%%%%%%%%%%%%%%%%%%%%%%%%%%%

    \index{GroundSegment.models.Tle.Tle \textit{(class)}|(}
\subsection{Class Tle}

    \label{GroundSegment:models:Tle:Tle}
\begin{tabular}{cccccc}
% Line for django.db.models.Model, linespec=[False]
\multicolumn{2}{r}{\settowidth{\BCL}{django.db.models.Model}\multirow{2}{\BCL}{django.db.models.Model}}
&&
  \\\cline{3-3}
  &&\multicolumn{1}{c|}{}
&&
  \\
&&\multicolumn{2}{l}{\textbf{GroundSegment.models.Tle.Tle}}
\end{tabular}

\begin{alltt}

Clase/Entidad TLE.
Almacena la informacion de los TLE incluyendo su fecha de descarga y la epoca de TLE 
\end{alltt}


%%%%%%%%%%%%%%%%%%%%%%%%%%%%%%%%%%%%%%%%%%%%%%%%%%%%%%%%%%%%%%%%%%%%%%%%%%%
%%                                Methods                                %%
%%%%%%%%%%%%%%%%%%%%%%%%%%%%%%%%%%%%%%%%%%%%%%%%%%%%%%%%%%%%%%%%%%%%%%%%%%%

  \subsubsection{Methods}

    \label{GroundSegment:models:Tle:Tle:getLine1}
    \index{GroundSegment.models.Tle.Tle \textit{(class)}!GroundSegment.models.Tle.Tle.getLine1 \textit{(method)}}

    \vspace{0.5ex}

\hspace{.8\funcindent}\begin{boxedminipage}{\funcwidth}

    \raggedright \textbf{getLine1}(\textit{self})

    \vspace{-1.5ex}

    \rule{\textwidth}{0.5\fboxrule}
\setlength{\parskip}{2ex}
\begin{alltt}

Retorma la primera linea del TLE en formato texto plano
@rtype:   string
@return:  primera linea del TLE en texto plano.
\end{alltt}

\setlength{\parskip}{1ex}
    \end{boxedminipage}

    \label{GroundSegment:models:Tle:Tle:getLine2}
    \index{GroundSegment.models.Tle.Tle \textit{(class)}!GroundSegment.models.Tle.Tle.getLine2 \textit{(method)}}

    \vspace{0.5ex}

\hspace{.8\funcindent}\begin{boxedminipage}{\funcwidth}

    \raggedright \textbf{getLine2}(\textit{self})

    \vspace{-1.5ex}

    \rule{\textwidth}{0.5\fboxrule}
\setlength{\parskip}{2ex}
\begin{alltt}

Retorma la segunda linea del TLE en formato texto plano
@rtype:   string
@return:  segunda linea del TLE en texto plano.
\end{alltt}

\setlength{\parskip}{1ex}
    \end{boxedminipage}


%%%%%%%%%%%%%%%%%%%%%%%%%%%%%%%%%%%%%%%%%%%%%%%%%%%%%%%%%%%%%%%%%%%%%%%%%%%
%%                            Class Variables                            %%
%%%%%%%%%%%%%%%%%%%%%%%%%%%%%%%%%%%%%%%%%%%%%%%%%%%%%%%%%%%%%%%%%%%%%%%%%%%

  \subsubsection{Class Variables}

    \vspace{-1cm}
\hspace{\varindent}\begin{longtable}{|p{\varnamewidth}|p{\vardescrwidth}|l}
\cline{1-2}
\cline{1-2} \centering \textbf{Name} & \centering \textbf{Description}& \\
\cline{1-2}
\endhead\cline{1-2}\multicolumn{3}{r}{\small\textit{continued on next page}}\\\endfoot\cline{1-2}
\endlastfoot\raggedright t\-l\-e\-D\-a\-t\-e\-T\-i\-m\-e\- & \raggedright \begin{alltt}

Fecha generacion del TLE, si no fuera seteada se utilizara la fecha hora actual
\end{alltt}

\textbf{Value:} 
{\tt models.DateTimeField(auto\_now\_add= True)}&\\
\cline{1-2}
\raggedright d\-o\-w\-n\-l\-o\-a\-d\-e\-d\- & \raggedright \begin{alltt}

Fecha de descarga del TLE
\end{alltt}

\textbf{Value:} 
{\tt models.DateTimeField(auto\_now\_add= True)}&\\
\cline{1-2}
\raggedright l\-i\-n\-e\-s\- & \raggedright \begin{alltt}

Lineas del TLE
\end{alltt}

\textbf{Value:} 
{\tt models.TextField(max\_length= 124,)}&\\
\cline{1-2}
\raggedright s\-a\-t\-e\-l\-l\-i\-t\-e\- & \raggedright \begin{alltt}

Satelite asociado al TLE
\end{alltt}

\textbf{Value:} 
{\tt models.ForeignKey(Satellite, related\_name= 'tles')}&\\
\cline{1-2}
\end{longtable}

    \index{GroundSegment.models.Tle.Tle \textit{(class)}|)}
    \index{GroundSegment \textit{(package)}!GroundSegment.models \textit{(package)}!GroundSegment.models.Tle \textit{(module)}|)}
