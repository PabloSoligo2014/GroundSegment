%
% API Documentation for GroundSegment Technical DataSheet
% Module GroundSegment.models.Notification.MessageTemplate
%
% Generated by epydoc 3.0.1
% [Mon Dec  5 13:57:26 2016]
%

%%%%%%%%%%%%%%%%%%%%%%%%%%%%%%%%%%%%%%%%%%%%%%%%%%%%%%%%%%%%%%%%%%%%%%%%%%%
%%                          Module Description                           %%
%%%%%%%%%%%%%%%%%%%%%%%%%%%%%%%%%%%%%%%%%%%%%%%%%%%%%%%%%%%%%%%%%%%%%%%%%%%

    \index{GroundSegment \textit{(package)}!GroundSegment.models \textit{(package)}!GroundSegment.models.Notification \textit{(package)}!GroundSegment.models.Notification.MessageTemplate \textit{(module)}|(}
\section{Module GroundSegment.models.Notification.MessageTemplate}

    \label{GroundSegment:models:Notification:MessageTemplate}
Created on Sep 27, 2016

\textbf{Author:} ubuntumate




%%%%%%%%%%%%%%%%%%%%%%%%%%%%%%%%%%%%%%%%%%%%%%%%%%%%%%%%%%%%%%%%%%%%%%%%%%%
%%                           Class Description                           %%
%%%%%%%%%%%%%%%%%%%%%%%%%%%%%%%%%%%%%%%%%%%%%%%%%%%%%%%%%%%%%%%%%%%%%%%%%%%

    \index{GroundSegment \textit{(package)}!GroundSegment.models \textit{(package)}!GroundSegment.models.Notification \textit{(package)}!GroundSegment.models.Notification.MessageTemplate \textit{(module)}!GroundSegment.models.Notification.MessageTemplate.MessageTemplate \textit{(class)}|(}
\subsection{Class MessageTemplate}

    \label{GroundSegment:models:Notification:MessageTemplate:MessageTemplate}
\begin{tabular}{cccccc}
% Line for django.db.models.Model, linespec=[False]
\multicolumn{2}{r}{\settowidth{\BCL}{django.db.models.Model}\multirow{2}{\BCL}{django.db.models.Model}}
&&
  \\\cline{3-3}
  &&\multicolumn{1}{c|}{}
&&
  \\
&&\multicolumn{2}{l}{\textbf{GroundSegment.models.Notification.MessageTemplate.MessageTemplate}}
\end{tabular}

\begin{alltt}

Plantilla para la notificacion, al momento de enviar una el mensaje se toma el texto de la plantilla
se realizan los remplazos por la information asociada (delimitada entre [])  y se genera el texto final.

Ej:

    Se ha generado la alarma numero [pk] a hora abordo [dtOnBoard] y es del tipo [alarmaType.code]...etc
\end{alltt}


%%%%%%%%%%%%%%%%%%%%%%%%%%%%%%%%%%%%%%%%%%%%%%%%%%%%%%%%%%%%%%%%%%%%%%%%%%%
%%                                Methods                                %%
%%%%%%%%%%%%%%%%%%%%%%%%%%%%%%%%%%%%%%%%%%%%%%%%%%%%%%%%%%%%%%%%%%%%%%%%%%%

  \subsubsection{Methods}

    \label{GroundSegment:models:Notification:MessageTemplate:MessageTemplate:__str__}
    \index{GroundSegment \textit{(package)}!GroundSegment.models \textit{(package)}!GroundSegment.models.Notification \textit{(package)}!GroundSegment.models.Notification.MessageTemplate \textit{(module)}!GroundSegment.models.Notification.MessageTemplate.MessageTemplate \textit{(class)}!GroundSegment.models.Notification.MessageTemplate.MessageTemplate.\_\_str\_\_ \textit{(method)}}

    \vspace{0.5ex}

\hspace{.8\funcindent}\begin{boxedminipage}{\funcwidth}

    \raggedright \textbf{\_\_str\_\_}(\textit{self})

\setlength{\parskip}{2ex}
\setlength{\parskip}{1ex}
    \end{boxedminipage}


%%%%%%%%%%%%%%%%%%%%%%%%%%%%%%%%%%%%%%%%%%%%%%%%%%%%%%%%%%%%%%%%%%%%%%%%%%%
%%                            Class Variables                            %%
%%%%%%%%%%%%%%%%%%%%%%%%%%%%%%%%%%%%%%%%%%%%%%%%%%%%%%%%%%%%%%%%%%%%%%%%%%%

  \subsubsection{Class Variables}

    \vspace{-1cm}
\hspace{\varindent}\begin{longtable}{|p{\varnamewidth}|p{\vardescrwidth}|l}
\cline{1-2}
\cline{1-2} \centering \textbf{Name} & \centering \textbf{Description}& \\
\cline{1-2}
\endhead\cline{1-2}\multicolumn{3}{r}{\small\textit{continued on next page}}\\\endfoot\cline{1-2}
\endlastfoot\raggedright n\-a\-m\-e\- & \raggedright \textbf{Value:} 
{\tt models.CharField('Nombre de la plantilla', max\_length= Co\texttt{...}}&\\
\cline{1-2}
\raggedright s\-u\-b\-j\-e\-c\-t\- & \raggedright \textbf{Value:} 
{\tt models.CharField('Asunto del mensaje', max\_length= Consts\texttt{...}}&\\
\cline{1-2}
\raggedright t\-e\-x\-t\- & \raggedright \textbf{Value:} 
{\tt models.TextField(null= False, default= "Sin mensaje")}&\\
\cline{1-2}
\end{longtable}

    \index{GroundSegment \textit{(package)}!GroundSegment.models \textit{(package)}!GroundSegment.models.Notification \textit{(package)}!GroundSegment.models.Notification.MessageTemplate \textit{(module)}!GroundSegment.models.Notification.MessageTemplate.MessageTemplate \textit{(class)}|)}
    \index{GroundSegment \textit{(package)}!GroundSegment.models \textit{(package)}!GroundSegment.models.Notification \textit{(package)}!GroundSegment.models.Notification.MessageTemplate \textit{(module)}|)}
