%
% API Documentation for GroundSegment Technical DataSheet
% Module GroundSegment.models.CommandType
%
% Generated by epydoc 3.0.1
% [Tue Sep 27 17:02:06 2016]
%

%%%%%%%%%%%%%%%%%%%%%%%%%%%%%%%%%%%%%%%%%%%%%%%%%%%%%%%%%%%%%%%%%%%%%%%%%%%
%%                          Module Description                           %%
%%%%%%%%%%%%%%%%%%%%%%%%%%%%%%%%%%%%%%%%%%%%%%%%%%%%%%%%%%%%%%%%%%%%%%%%%%%

    \index{GroundSegment \textit{(package)}!GroundSegment.models \textit{(package)}!GroundSegment.models.CommandType \textit{(module)}|(}
\section{Module GroundSegment.models.CommandType}

    \label{GroundSegment:models:CommandType}
\begin{alltt}

Created on 25 de ago. de 2016

@author: pabli
\end{alltt}


%%%%%%%%%%%%%%%%%%%%%%%%%%%%%%%%%%%%%%%%%%%%%%%%%%%%%%%%%%%%%%%%%%%%%%%%%%%
%%                           Class Description                           %%
%%%%%%%%%%%%%%%%%%%%%%%%%%%%%%%%%%%%%%%%%%%%%%%%%%%%%%%%%%%%%%%%%%%%%%%%%%%

    \index{GroundSegment.models.CommandType.CommandType \textit{(class)}|(}
\subsection{Class CommandType}

    \label{GroundSegment:models:CommandType:CommandType}
\begin{tabular}{cccccc}
% Line for django.db.models.Model, linespec=[False]
\multicolumn{2}{r}{\settowidth{\BCL}{django.db.models.Model}\multirow{2}{\BCL}{django.db.models.Model}}
&&
  \\\cline{3-3}
  &&\multicolumn{1}{c|}{}
&&
  \\
&&\multicolumn{2}{l}{\textbf{GroundSegment.models.CommandType.CommandType}}
\end{tabular}


%%%%%%%%%%%%%%%%%%%%%%%%%%%%%%%%%%%%%%%%%%%%%%%%%%%%%%%%%%%%%%%%%%%%%%%%%%%
%%                                Methods                                %%
%%%%%%%%%%%%%%%%%%%%%%%%%%%%%%%%%%%%%%%%%%%%%%%%%%%%%%%%%%%%%%%%%%%%%%%%%%%

  \subsubsection{Methods}

    \label{GroundSegment:models:CommandType:CommandType:__str__}
    \index{GroundSegment.models.CommandType.CommandType \textit{(class)}!GroundSegment.models.CommandType.CommandType.\_\_str\_\_ \textit{(method)}}

    \vspace{0.5ex}

\hspace{.8\funcindent}\begin{boxedminipage}{\funcwidth}

    \raggedright \textbf{\_\_str\_\_}(\textit{self})

\setlength{\parskip}{2ex}
\setlength{\parskip}{1ex}
    \end{boxedminipage}


%%%%%%%%%%%%%%%%%%%%%%%%%%%%%%%%%%%%%%%%%%%%%%%%%%%%%%%%%%%%%%%%%%%%%%%%%%%
%%                            Class Variables                            %%
%%%%%%%%%%%%%%%%%%%%%%%%%%%%%%%%%%%%%%%%%%%%%%%%%%%%%%%%%%%%%%%%%%%%%%%%%%%

  \subsubsection{Class Variables}

    \vspace{-1cm}
\hspace{\varindent}\begin{longtable}{|p{\varnamewidth}|p{\vardescrwidth}|l}
\cline{1-2}
\cline{1-2} \centering \textbf{Name} & \centering \textbf{Description}& \\
\cline{1-2}
\endhead\cline{1-2}\multicolumn{3}{r}{\small\textit{continued on next page}}\\\endfoot\cline{1-2}
\endlastfoot\raggedright c\-o\-d\-e\- & \raggedright \textbf{Value:} 
{\tt models.CharField('Codigo del tipo de comando', max\_length\texttt{...}}&\\
\cline{1-2}
\raggedright d\-e\-s\-c\-r\-i\-p\-t\-i\-o\-n\- & \raggedright \textbf{Value:} 
{\tt models.CharField('Decripcion del tipo de comando', max\_le\texttt{...}}&\\
\cline{1-2}
\raggedright s\-a\-t\-e\-l\-l\-i\-t\-e\- & \raggedright \textbf{Value:} 
{\tt models.ForeignKey(Satellite, related\_name= 'commandsType')}&\\
\cline{1-2}
\raggedright s\-a\-t\-e\-l\-l\-i\-t\-e\-S\-t\-a\-t\-e\-s\- & \raggedright \textbf{Value:} 
{\tt models.ManyToManyField(SatelliteState, related\_name= 'com\texttt{...}}&\\
\cline{1-2}
\raggedright a\-c\-t\-i\-v\-e\- & \raggedright \textbf{Value:} 
{\tt models.BooleanField(default= True)}&\\
\cline{1-2}
\raggedright t\-r\-a\-n\-s\-a\-c\-t\-i\-o\-n\-a\-l\- & \raggedright \textbf{Value:} 
{\tt models.BooleanField(default= False)}&\\
\cline{1-2}
\raggedright t\-i\-m\-e\-o\-u\-t\- & \raggedright \textbf{Value:} 
{\tt models.IntegerField('Tiempo en segundos?', default= 0, nu\texttt{...}}&\\
\cline{1-2}
\raggedright n\-o\-t\-e\-s\- & \raggedright \textbf{Value:} 
{\tt models.TextField('Consecuencias, restricciones del comand\texttt{...}}&\\
\cline{1-2}
\end{longtable}

    \index{GroundSegment.models.CommandType.CommandType \textit{(class)}|)}
    \index{GroundSegment \textit{(package)}!GroundSegment.models \textit{(package)}!GroundSegment.models.CommandType \textit{(module)}|)}
